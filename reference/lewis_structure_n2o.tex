\documentclass{article}
\usepackage[utf8]{inputenc}
\usepackage{amsmath}
\usepackage{amssymb}
\usepackage{amsfonts}
\usepackage{mathtools}
\usepackage{chemmacros}
\chemsetup{
    modules = all,
}
\usepackage{chemfig}
\setchemfig{%
    % scheme debug=true,% set to false in your document
    fixed length=true,
    atom sep=2em,
}
\newcommand{\bndlen}{2.0em}
\newcommand{\lnwidth}{0.7pt}

\usepackage[version=4]{mhchem}

\title{Why \ce{N2O} is not symmetry}
\author{dimsplendid }
\date{April 2022}

\begin{document}

\maketitle

\section{Introduction}

\subsection{\ce{N2O} resonance structure}

\schemestart
    \chemleft{[}
    \subscheme{
    % \setcharge{debug}
        \chemfig{
            @{n1}\charge{[circle]180=\:}{N}~[@{tb}]\chemabove{N}{\scriptstyle\oplus}-[@{sb}]@{o1}\charge{[circle]0=\:,90=\:, 270=\:, 45[circle,anchor=180+\chargeangle]=$\scriptstyle\ominus$}{O}
        }
        \arrow{<->}
        \chemfig{
            \charge{[circle]180=\:, 270=\:,90=$\scriptstyle\ominus$}{N}=\chemabove{N}{\scriptstyle\oplus}=\charge{[circle]0=\:,-90=\:}{O}
        }
    }
    \chemright{]}
\schemestop
\chemmove[red, -stealth, shorten <=3pt]{
    \draw(tb).. controls +(-90:.4cm) and +(-75:4mm).. (n1);
    \draw(o1).. controls +(120:.4cm) and +(90:4mm).. (sb);
}
\\
\\
\\
\schemestart
    \chemleft{[}
    \subscheme{
    % \setcharge{debug}
        \chemfig{
            \charge{[circle]180=\:,-90=\:,90=$\scriptstyle\ominus$}{N}=\chemabove{O}{2\scriptstyle\oplus}=\charge{[circle]0=\:,-90=\:,90=$\scriptstyle\ominus$}{N}
        }
    }
    \chemright{]}
\schemestop
\end{document}
